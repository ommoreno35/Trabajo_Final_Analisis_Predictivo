
\documentclass{beamer}
\usetheme{Madrid}
\usepackage[utf8]{inputenc}
\usepackage[spanish]{babel}
\usepackage{graphicx}
\usepackage{lmodern}
\usepackage{tikz}

\title{Análisis Predictivo y Gestión de Datos}
\subtitle{Sesión 2: Preparación y Limpieza de Datos}
\author{Oscar Leonardo Rincón León}
\date{\today}

\begin{document}

\frame{\titlepage}

% Objetivos
\begin{frame}{Objetivos de la sesión}
\begin{itemize}
    \item Comprender cómo la calidad de los datos afecta la capacidad predictiva
    \item Identificar problemas comunes en los datos y su impacto en los modelos
    \item Conectar la limpieza con la etapa de preparación de datos en CRISP-DM
    \item Aplicar estrategias prácticas de depuración y transformación
\end{itemize}
\end{frame}

% Importancia de la preparación
\begin{frame}{¿Por qué importa la preparación en análisis predictivo?}
\begin{itemize}
    \item Los modelos aprenden patrones: si los datos están distorsionados, aprenderán mal
    \item Variables mal estructuradas → patrones falsos
    \item Ruido = reducción del poder predictivo
    \item La limpieza es una forma de modelar: decidir qué mantener y cómo
\end{itemize}
\end{frame}


% Introducción a CRISP-DM
\begin{frame}{¿Qué es CRISP-DM?}
\begin{itemize}
    \item \textbf{CRISP-DM} significa Cross-Industry Standard Process for Data Mining.
    \item Es un modelo estándar y ampliamente adoptado para desarrollar proyectos de ciencia de datos.
    \item Proporciona una estructura organizada para abordar problemas de negocio utilizando datos.
\end{itemize}
\end{frame}

% CRISP-DM como metodología general
\begin{frame}{¿Por qué usar CRISP-DM?}
\begin{itemize}
    \item Es aplicable en múltiples sectores: salud, educación, industria, gobierno.
    \item Su enfoque cíclico permite iterar, corregir y mejorar continuamente.
    \item Ayuda a comunicar el proceso de análisis entre equipos multidisciplinarios.
    \item Aporta una guía paso a paso para estructurar el trabajo analítico desde la comprensión del problema hasta la implementación.
\end{itemize}
\end{frame}



% Etapas del ciclo CRISP-DM – Parte 1
\begin{frame}{Etapas del ciclo CRISP-DM (1/2)}
\begin{itemize}
    \item \textbf{1. Comprensión del contexto:} entender el problema, sus objetivos y su impacto en un área como educación, salud o administración pública.
    \item \textbf{2. Comprensión de los datos:} recolectar, explorar y validar la calidad de los datos disponibles.
    \item \textbf{3. Preparación de los datos:} limpiar, transformar, codificar y seleccionar variables útiles para el análisis posterior.
\end{itemize}
\end{frame}

% Etapas del ciclo CRISP-DM – Parte 2
\begin{frame}{Etapas del ciclo CRISP-DM (2/2)}
\begin{itemize}
    \item \textbf{4. Modelado:} seleccionar y entrenar modelos estadísticos o de aprendizaje automático.
    \item \textbf{5. Evaluación:} verificar si el modelo resuelve el problema, usando métricas adecuadas y revisando supuestos.
    \item \textbf{6. Despliegue:} comunicar resultados, implementar soluciones o integrarlas en procesos de decisión.
\end{itemize}
\end{frame}


% CRISP-DM y preparación de datos
\begin{frame}{Preparación en el ciclo CRISP-DM}
\begin{itemize}
    \item La preparación de datos es la \textbf{tercera etapa} del ciclo
    \item Asegura que los datos estén limpios, completos y listos para el modelado
    \item Impacta directamente en:
    \begin{itemize}
        \item Selección adecuada de modelos
        \item Métricas válidas de evaluación
        \item Interpretación clara de resultados
    \end{itemize}
\end{itemize}
\end{frame}

% Decisiones clave
\begin{frame}{Decisiones clave al preparar datos}
\begin{itemize}
    \item ¿Qué hacer con valores faltantes?
    \item ¿Qué variables eliminar?
    \item ¿Qué transformar?
    \item ¿Qué codificar o escalar?
    \item Cada decisión afecta la precisión, estabilidad e interpretabilidad del modelo
\end{itemize}
\end{frame}

% Errores comunes
\begin{frame}{Errores comunes y sus consecuencias}
\begin{itemize}
    \item Mantener variables con alta proporción de nulos → modelos inconsistentes
    \item No escalar variables → sesgo en modelos sensibles a magnitudes (KNN, SVM)
    \item Duplicados → sobreajuste
    \item Codificación incorrecta → distorsión de relaciones
\end{itemize}
\end{frame}

% Herramientas en Python

% Fundamentos teóricos sobre preparación de datos
\begin{frame}{¿Por qué es crucial la preparación de los datos?}
\begin{itemize}
    \item La calidad de los datos condiciona la calidad de los modelos predictivos.
    \item Los modelos aprenden patrones de los datos disponibles: si hay errores, aprenden mal.
    \item En proyectos reales, más del 70\% del tiempo se invierte en preparación, no modelado.
\end{itemize}
\end{frame}

\begin{frame}{Preparación en el ciclo CRISP-DM}
\begin{itemize}
    \item Es la tercera etapa del ciclo y conecta la comprensión del negocio con la implementación del modelo.
    \item Transforma datos crudos en una base coherente y lista para el análisis.
    \item Afecta directamente la calidad de los resultados, su interpretación y utilidad práctica.
\end{itemize}
\end{frame}

\begin{frame}{Errores comunes a corregir}
\begin{itemize}
    \item \textbf{Valores faltantes:} eliminación o imputación (media, mediana, por grupo).
    \item \textbf{Variables mal codificadas:} ordinales, categóricas, numéricas que no lo son.
    \item \textbf{Escalas incompatibles:} requiere escalado para evitar sesgos.
    \item \textbf{Outliers:} afectan modelos sensibles; deben detectarse y analizarse.
\end{itemize}
\end{frame}

\begin{frame}{Preparación y aprendizaje supervisado}
\begin{itemize}
    \item En aprendizaje supervisado, todo modelo depende de una buena relación entre variables explicativas ($X$) y objetivo ($Y$).
    \item Si los datos no están bien preparados, los modelos no aprenderán patrones útiles.
    \item La limpieza garantiza interpretabilidad, generalización y confianza en los resultados.
\end{itemize}
\end{frame}

\begin{frame}{Referencias clave}
\begin{itemize}
    \item \textbf{Hands-On ML (Cap. 2):} destaca la limpieza y transformación como parte esencial del flujo ML.
    \item \textbf{ISLP:} enfatiza que un análisis válido depende de datos estructurados y bien documentados.
    \item \textbf{CRISP-DM:} define la preparación como puente entre exploración y modelado.
\end{itemize}
\end{frame}

\begin{frame}{Preparación práctica con Python}
\begin{itemize}
    \item \texttt{pandas}: limpieza, filtrado, imputación
    \item \texttt{numpy}: operaciones matemáticas y máscaras
    \item \texttt{scikit-learn}: escalado, codificación, pipelines
\end{itemize}
\end{frame}

% Tratamiento de nulos
\begin{frame}{Tratamiento de valores faltantes}
\begin{itemize}
    \item Opciones:
    \begin{itemize}
        \item Eliminación: si son pocos y aleatorios
        \item Imputación: media, mediana, moda, por grupos
        \item Modelos de imputación (avanzado)
    \end{itemize}
    \item Toda imputación introduce supuestos
\end{itemize}
\end{frame}

% Codificación
\begin{frame}{Codificación de variables categóricas}
\begin{itemize}
    \item Convertir texto en números para modelos
    \item Opciones:
    \begin{itemize}
        \item Ordinal: cuando hay jerarquía
        \item One-hot encoding: para evitar supuestos
    \end{itemize}
    \item Evitar codificación que induzca relaciones falsas
\end{itemize}
\end{frame}

% Escalado
\begin{frame}{Escalado de variables numéricas}
\begin{itemize}
    \item Modelos como KNN y regresión logística son sensibles a magnitudes
    \item Opciones:
    \begin{itemize}
        \item Min-Max Scaling
        \item Z-score (estandarización)
    \end{itemize}
    \item Escalar antes de entrenar y validar, nunca después
\end{itemize}
\end{frame}

% Ejercicio práctico
\begin{frame}{Ejercicio práctico}
\begin{itemize}
    \item Dataset con problemas reales:
    \begin{itemize}
        \item Valores nulos
        \item Categóricas no codificadas
        \item Magnitudes no comparables
    \end{itemize}
    \item Meta: preparar para su uso en un modelo de clasificación binaria
\end{itemize}
\end{frame}

% Cierre
\begin{frame}{Cierre de la sesión}
\begin{itemize}
    \item La preparación no es técnica: es estratégica
    \item Afecta todo el proceso de modelado
    \item Un analista predictivo prepara pensando en el modelo, el error y la interpretación
\end{itemize}
\vspace{0.5cm}
\textbf{Tarea:} Aplicar limpieza a un nuevo dataset. Documentar decisiones y justificar imputaciones.
\end{frame}

\end{document}
